\documentclass[a4paper,11pt]{article}
\usepackage{luatex85}
\usepackage[babelshorthands]{polyglossia}
\usepackage[fleqn,reqno]{amsmath}
\usepackage{amsthm}
\RequirePackage[
	backend=biber,
	bibstyle=gost-numeric,
	citestyle=gost-numeric,
	%defernumbers=true,
	defernumbers=false,
	language=auto,
	autolang=langname,
	]{biblatex}
\usepackage[math-style=ISO,bold-style=upright,partial=italic]{unicode-math}
\usepackage{microtype}
\usepackage[width=16cm,height=24cm]{geometry}
\usepackage[russian]{hyperref}

\setmainlanguage{russian}

\setmainfont{Cambria}
\setsansfont{Calibri}
\setmonofont{Source Code Pro}[Scale=MatchLowercase]
\setmonofont{Source Code Pro}[Scale=MatchLowercase]
\setmathfont{Cambria Math}[sans-style=literal]
\setmathfont{XITS Math}[range={\mathscr}]

\addbibresource{\jobname.bib}

\makeatletter

\allowdisplaybreaks[4]

\def\[#1\]{\begin{align*}#1\end{align*}}
\newcommand\eqtag[1]{\refstepcounter{equation}\tag{\theequation}\label{#1}}

\title{Алгебра Ли вариационных симметрий\\лагранжевой задачи}
\author{В. Боровик}




\begin{document}

\maketitle

Пусть $M$~— гладкое многообразие с локальными координатами $x~=~(x_1,...,x_n)$,
$\mathbf{v}=\sum_{i=1}^n\xi_i(x)\partial_{x_{i}}$~— векторное поле на нем. Оно
является \textit{полем обычной вариационной симметрии} лагранжевой задачи
$\mathcal{L}[x]=\int L(t,x,\dot{x})\,dt$, если 
	\[
	\mathbf{pr}^{(1)}\mathbf{v}(L) = \sum_{i=1}^n \xi_i \partial_{x_{i}}(L) + \sum_{i=1}^n \dot{\xi_i} \partial_{\dot{x_{i}}}(L) = 0.
	\eqtag{e1}
	\]
Введем скобку Ли на пространстве векторных полей:
$[\mathbf{v},\mathbf{w}]=\mathbf{v} \circ \mathbf{w} - \mathbf{w} \circ
\mathbf{v}$. Проверим, что мы получили дифференциальный оператор первого
порядка, то есть можем его отождествить с~векторным полем.

Пусть $\mathbf{v} = \sum_{i=1}^{n} \xi_i(x) \partial_{x_{i}}$,
$\mathbf{w} = \sum_{j=1}^{n} \eta_j(x) \partial_{x_{j}}$. Тогда
	\[
	&(\mathbf{v} \circ \mathbf{w} - \mathbf{w} \circ \mathbf{v}) (f) = \sum_{i=1}^n \xi_i \partial_{x_i} \left( \sum_{j=1}^{n} \eta_j \partial_{x_{j}}(f) \right) - \sum_{i=1}^n \eta_i \partial_{x_i} \left( \sum_{j=1}^{n} \xi_j \partial_{x_{j}}(f) \right) ={}\\
	&\qquad
		=\sum_{i = 1}^n \sum_{j=1}^n \left[ \xi_i \partial_{x_i}(\eta_j)  \partial_{x_j}(f) + \xi_i \eta_j \frac{\partial (f)}{\partial x_i \partial x_j} - \left(\eta_i \partial_{x_i}(\xi_j)  \partial_{x_j}(f) + \eta_i \xi_j \frac{\partial (f)}{\partial x_i \partial x_j} \right) \right]={}\\
	&\qquad
		=\sum_{i = 1}^n \sum_{j=1}^n \left( \xi_i \partial_{x_i}(\eta_j)  \partial_{x_j}(f) - \eta_i \partial_{x_i}(\xi_j)  \partial_{x_j}(f)  \right) + \sum_{i = 1}^n \sum_{j=1}^n \left(  \xi_i \eta_j \frac{\partial (f)}{\partial x_i \partial x_j} - \eta_i \xi_j \frac{\partial (f)}{\partial x_i \partial x_j}  \right),
	\]
причем последняя двойная сумма равна нулю: 
	\[
	\sum_{i = 1}^n \sum_{j=1}^n \left(  \xi_i \eta_j \frac{\partial (f)}{\partial x_i \partial x_j} - \eta_i \xi_j \frac{\partial (f)}{\partial x_i \partial x_j}  \right) =\sum_{i = 1}^n \sum_{j=1}^n  \xi_i \eta_j \frac{\partial (f)}{\partial x_i \partial x_j} - \sum_{i = 1}^n \sum_{j=1}^n \eta_i \xi_j \frac{\partial (f)}{\partial x_i \partial x_j} = 0.
	\]
Отсюда
	\[
	(\mathbf{v} \circ \mathbf{w} - \mathbf{w} \circ \mathbf{v})(f) = \sum_{j = 1}^n \left[ \sum_{i=1}^n \left( \xi_i \partial_{x_i}(\eta_j)   - \eta_i \partial_{x_i}(\xi_j) \right)\right]  \partial_{x_j}(f),
	\]
следовательно,
	\[
	\mathbf{v} \circ \mathbf{w} - \mathbf{w} \circ \mathbf{v} = \sum_{j = 1}^n \zeta_j \partial_{x_j}, \mbox{где} \ \ \zeta_j = \sum_{i=1}^n \left( \xi_i \partial_{x_i}(\eta_j)   - \eta_i \partial_{x_i}(\xi_j) \right) = \mathbf{v}(\eta_j) - \mathbf{w}(\xi_j).
	\]

Кососимметричность такой скобки очевидна, проверим тождество Якоби. Пусть $\mathbf{u} = \sum_{i=1}^{n} \tau_i(x) \partial_{x_{i}}$, тогда:
	\[
	&[\mathbf{u}, [\mathbf{v},\mathbf{w}]]
		=\sum_{j=1}^{n} \mathbf{u} (\mathbf{v}(\eta_j) - \mathbf{w}(\xi_j)) -  [\mathbf{v},\mathbf{w}](\tau_j)={}\\
	&\qquad
		=\sum_{j=1}^{n} \mathbf{u} \circ \mathbf{v}(\eta_j)  - \mathbf{u} \circ \mathbf{w}(\xi_j) - \mathbf{v} \circ \mathbf{w}(\tau_j) + \mathbf{w} \circ \mathbf{v}(\tau_j).
	\]
Аналогично,
	\[
	&[\mathbf{v}, [\mathbf{w},\mathbf{u}]]  = \sum_{j=1}^{n} \mathbf{v} \circ \mathbf{w}(\tau_j)  - \mathbf{v} \circ \mathbf{u}(\eta_j) - \mathbf{w} \circ \mathbf{u}(\xi_j) + \mathbf{u} \circ \mathbf{w}(\xi_j),\\
	&[\mathbf{w}, [\mathbf{u},\mathbf{v}]]  = \sum_{j=1}^{n} \mathbf{w} \circ \mathbf{u}(\xi_j)  - \mathbf{w} \circ \mathbf{v}(\tau_j) - \mathbf{u} \circ \mathbf{v}(\eta_j) + \mathbf{v} \circ \mathbf{u}(\eta_j).
	\]
Следовательно, $[\mathbf{u}, [\mathbf{v},\mathbf{w}]] + [\mathbf{v},
[\mathbf{w},\mathbf{u}]] + [\mathbf{w}, [\mathbf{u},\mathbf{v}]] = 0$, и
пространство векторных полей на многообразии с операцией коммутирования
образует алгебру Ли.

Теперь достаточно проверить, что подпространство полей вариационных симметрий
инвариантно относительно введенной скобки. Пусть $\mathbf{v}, \mathbf{w}$~—
поля вариационных симметрий, то есть они удовлетворяют~\eqref{e1}. Проверим,
что $[\mathbf{v}, \mathbf{w}]$  тоже удовлетворяет~\eqref{e1}.

Операция первого продолжения отображает однопараметрическую локальную группу Ли
$G=\{g^\epsilon\}$ преобразований $M$ в однопараметрическую локальную группу Ли
$\mathbf{pr}^{(1)}G$ преобразований $M^*=\{(x,\dot{x})\}$. Дифференциал этого
отображения сопоставляет векторному полю $\mathbf{v}$ на $M$ его продолжение
$\mathbf{pr}^{(1)}\mathbf{v}$ и~стреляет из алгебры Ли группы~$G$ в~алгебру Ли
группы $\mathbf{pr}^{(1)}G$. В~силу функториальности дифференциала, имеем:
	\[
	\mathbf{pr}^{(1)}[\mathbf{v}, \mathbf{w}] = [\mathbf{pr}^{(1)}\mathbf{v}, \mathbf{pr}^{(1)}\mathbf{w}].
	\]

\if 0
С одной стороны,

$\mathbf{pr}^{(1)}[\mathbf{v}, \mathbf{w}](f) = \sum_{j=1}^n \zeta_j \partial_{x_{j}}(f) + \sum_{j=1}^n \dot{\zeta_j} \partial_{\dot{x_{j}}}(f) = \sum_{j=1}^n \sum_{i=1}^n \left( \xi_i \partial_{x_i}(\eta_j)   - \eta_i \partial_{x_i}(\xi_j) \right)\partial_{x_{j}}(f) + \sum_{j=1}^n \sum_{i=1}^n \left( \dot{\xi_i} \partial_{x_i}(\eta_j) - \dot{\eta_i} \partial_{x_i}(\xi_j) + \xi_i \frac{d}{dt}(\partial_{x_i}(\eta_j)) - \eta_i \frac{d}{dt}(\partial_{x_i}(\xi_j)) \right) \partial_{\dot{x_{j}}}(f)$

\fi
Тогда $$\mathbf{pr}^{(1)}[\mathbf{v}, \mathbf{w}](L) = (\mathbf{pr}^{(1)}\mathbf{v} \circ  \mathbf{pr}^{(1)}\mathbf{w}) (L) - (\mathbf{pr}^{(1)}\mathbf{w} \circ  \mathbf{pr}^{(1)}\mathbf{v}) (L) = \mathbf{pr}^{(1)}\mathbf{v}(0) - \mathbf{pr}^{(1)}\mathbf{w}(0) = 0.$$
Таким образом, $[\mathbf{v}, \mathbf{w}]$~— снова поле вариационной симметрии, а значит подпространство полей вариационных симметрий~— алгебра Ли.
\end{document}
