\documentclass[a4paper,11pt]{article}
\usepackage{luatex85}
\usepackage[babelshorthands]{polyglossia}
\usepackage[fleqn,reqno]{amsmath}
\usepackage{amsthm}
\RequirePackage[
	backend=biber,
	bibstyle=gost-numeric,
	citestyle=gost-numeric,
	%defernumbers=true,
	defernumbers=false,
	language=auto,
	autolang=langname,
	]{biblatex}
\usepackage[math-style=ISO,bold-style=upright,partial=italic]{unicode-math}
\usepackage{microtype}
\usepackage[width=16cm,height=24cm]{geometry}
\usepackage[russian]{hyperref}
\usepackage{luamplib}

\setmainlanguage{russian}

\setmainfont{Cambria}
\setsansfont{Calibri}
\setmonofont{Source Code Pro}[Scale=MatchLowercase]
\setmonofont{Source Code Pro}[Scale=MatchLowercase]
\setmathfont{Cambria Math}[sans-style=literal]
\setmathfont{XITS Math}[range={\mathscr,\leqslant,\geqslant}]

\addbibresource{\jobname.bib}

\makeatletter

\allowdisplaybreaks[4]

\def\[#1\]{\begin{align*}#1\end{align*}}
\newcommand\eqtag[1]{\refstepcounter{equation}\tag{\theequation}\label{#1}}

%\newcommand\slashfrac[2]{{#1\fracslash#2}}
\newcommand\slashfrac[2]{{#1/#2}}

\newcommand\pr{\operatorname{\textbf{\textup{pr}}}}

\theoremstyle{definition}
\newtheorem{theorem}{Теорема}
\newtheorem{lemma}{Лемма}
\newtheorem{definition}{Определение}
\newtheorem{example}{Пример}
%\newtheorem{proof}{Доказательство}

\newcommand\metasetup{\hypersetup{
	pdftitle=\@title,
	pdfauthor=\@author,
	linkbordercolor={0 .5 .25},
	}}

\setlength\overfullrule{5pt}

\makeatother

\begin{document}

\hyphenation{Ла-гран-жа ла-гран-же-вой}

\title{Канонические преобразования}
\author{А.~Н.~Швец}

\metasetup
\maketitle

\section{Канонические уравнения Гамильтона}

\emph{Канонические уравнения Гамильтона\/}
	\[
	\dot p=-\frac{\partial H}{\partial q},
	\quad
	\dot q=\frac{\partial H}{\partial p}
	\eqtag{eq:hamiltonian-eq}
	\]
служат альтернативой уравнениям Эйлера — Лагранжа в качестве уравнений движения
в~том случае, когда есть возможность найти \emph{функцию Гамильтона
(гамильтониан)}, выполнив преобразования Лежандра лагранжиана как функции
скоростей. Преобразование Лежандра возможно, если лагранжиан~— выпуклая функция
скоростей (проверьте!). Канонические уравнения представляют собой систему
$2n$ уравнений первого порядка ($n$~— число степеней свободы). Общий порядок
системы, естественно, тот~же, что и~у~уравнений Эйлера~— Лагранжа.

\section{Канонические преобразования}

Главнейший метод работы с~обыкновенными дифференциальными уравнениями~— замены
переменных, приводящие к упрощению или полному решению уравнений. Канонические
уравнения, в отличие от системы $2n$~уравнений первого порядка самого общего
вида, имеют очень специальный вид и обладают очень специальными важными
свойствами. Поэтому важны замены переменных (преобразования), сохраняющие
гамильтонову специфику канонических уравнений.

Мы рассмотрим преобразования, при которых осуществляется переход к новым
зависимым и независимой переменным следующего вида:
	\[
	(t,p,q)\mapsto(\hat t,\hat p,\hat q).
	\]

Под сохранением гамильтонова вида мы понимаем то, что после выполнения
преобразования в новых переменных уравнения примут эквивалентный вид
	\[
	\frac{d\hat p}{d\hat t}=-\frac{\partial\hat H}{\partial\hat q},
	\quad
	\frac{d\hat q}{d\hat t}=\frac{\partial\hat H}{\partial\hat p},
	\eqtag{eq:hamiltonian-eq-new}
	\]
где $\hat H(\hat t,\hat p,\hat q)$~— некоторая новая функция Гамильтона. Сразу
обращаем внимание, что новый гамильтониан, выраженный через старые переменные,
не обязан, вообще говоря, совпадать со~старым.

Оказывается, уравнения вида \eqref{eq:hamiltonian-eq} переходят в~уравнения вида
\eqref{eq:hamiltonian-eq-new}, если выполняется условие
	\[
	d\hat p\wedge d\hat q-d\hat H\wedge d\hat t=0
		\bmod dp\wedge dq-dH\wedge dt=0.
	\eqtag{eq:canon-cond-orig}
	\]
Здесь и~дальше по обозначению $dp\wedge dq=\sum_{i=1}^ndp_i\wedge dq_i$.
Согласно известной нам
лемме\footnote{См.~\url{http://mech.math.msu.su/~shvetz/coronavirus/IntrinsicForces.pdf}}
условие \eqref{eq:canon-cond-orig} превращается в~равносильное
	\[
	d\hat p\wedge d\hat q-d\hat H\wedge d\hat t
		=c(dp\wedge dq-dH\wedge dt).
	\eqtag{eq:canon-cond}
	\]
для некоторой постоянной~$c$. Преобразование с указанным свойством называется
\emph{каноническим,} а~число~$c$~— его \emph{валентностью.} При $c=1$ говорят
об~\emph{унивалентных\/} преобразованиях.

Условие \eqref{eq:canon-cond} можно понимать как условие замкнутости
$d\varphi=0$ внешней формы $\varphi=p\,dq-H\,dt-c(\hat p\,d\hat q-\hat H\,d\hat
t)$ (проверьте!). Согласно лемме Пуанкаре замкнутые внешние формы локально
точны (в более сильных формулировках, точны в выпуклых, или точны
в~\emph{звёздных\/} областях, то~есть в~таких, что содержат точку, из которой
просматривается каждая точка границы\footnote{Звёздность области требуется для
нахождения внешней первообразной для замкнутой формы в~соответствии с~так
называемой \emph{конической конструкцией Пуанкаре,} которая предполагает
гомотетическое стягивание области строго внутрь себя к~центру области.}).
Дальнейшие рассуждения имеют локальный характер.

Точная 1"=форма $\varphi$, таким образом, локально есть внешняя производная
некоторой 0"=формы $S$ (0"=формы отождествляются с функциями):
	\[
	p\,dq-H\,dt-c(\hat p\,d\hat q-\hat H\,d\hat t)=dS(t,\hat t,q,\hat q)
	\]
(функция $S$ зависит от тех букв, что стоят в~левой части под знаками
дифференциала). Приравнивая коэффициенты при одинаковых дифференциалах в левой
и правой частях, получим
	\[
	p=\frac{\partial S}{\partial q},
	\quad
	-c\hat p=\frac{\partial S}{\partial\hat q},
	\quad
	-H=\frac{\partial S}{\partial t},
	\quad
	c\hat H=\frac{\partial S}{\partial\hat t}.
	\]
В~дальнейшем ограничимся каноническими преобразованиями, сохраняющими время
($\hat t=t$). Тогда $S$ зависит от $(t,q,\hat q)$ и
	\[
	p=\frac{\partial S}{\partial q},
	\quad
	-c\hat p=\frac{\partial S}{\partial\hat q},
	\quad
	c\hat H-H=\frac{\partial S}{\partial t}.
	\eqtag{eq:canon-trans-1}
	\]
Полученные равенства связывают старые и~новые буквы, $(p,q,H)$ и~$(\hat p,\hat
q,\hat H)$, правда, в~виде, не разрешённом ни относительно старых, ни
относительно новых. Но так или иначе, $S$ содержит всю информацию
о~каноническом преобразовании, порождает его. Она называется
\emph{производящей функцией\/} канонического преобразования.

Увы, даже с~таким простым и очевидно каноническим преобразованием, как
тождественное, есть проблема. Поскольку при $\hat q=q$ должно быть
$\slashfrac{\partial S}{\partial\hat q}=\slashfrac{\partial S}{\partial q}$,
уравнения \eqref{eq:canon-trans-1} оказываются противоречивыми (убедитесь!).
Это плохо, так~как некоторые методы гамильтоновой механики предполагают
последовательное применение канонических преобразований, близких
к~тождественным, для последовательного упрощения канонических уравнений. Однако
представление канонического преобразования возможно при использовании
производящих функций других типов.

Можно в~форме~$\varphi$ выбрать некоторую группу дифференциальных мономов
и~проинтегрировать их по частям. К~примеру:
	\[
	\varphi=p\,dq+(c\hat H-H)\,dt-c\hat p\,d\hat q
		=p\,dq+(c\hat H-H)\,dt-c\,d(\hat p\hat q)+c\hat q\,d\hat p
		=dS,
	\]
или
	\[
	\tilde\varphi=p\,dq+(c\hat H-H)\,dt+c\hat q\,d\hat p
		=dS+c\,d(\hat p\hat q)=d\tilde S,
	\]
где $S=\tilde S(t,\hat p,q)=S+c\hat p\hat q$~— производящая функция другого
типа. Тогда
	\[
	p=\frac{\partial\tilde S}{\partial q},
	\quad
	c\hat q=\frac{\partial\tilde S}{\partial\hat p},
	\quad
	c\hat H-H=\frac{\partial\tilde S}{\partial t}.
	\eqtag{eq:canon-trans-1}
	\]
Разных типов производящих функций столько, сколькими способами выбираются
мономы для интегрирования по частям. Тождественное преобразование можно
представить с~помощью производящей функции типа~$\tilde S$ так, что уравнения
\eqref{eq:canon-trans-1} возможно разрешить относительно старых или
относительно новых переменных (проверьте!).

\section{Метод Якоби}

Все согласятся с тем, что простейший вид канонических уравнений достигается при
нулевом гамильтониане. Заманчиво было~бы найти каноническое преобразование,
после выполнения которого получилось~бы $\hat H(t,\hat p,\hat q)\equiv0$. Тогда
первое и~третье равенства \eqref{eq:canon-trans-1} дают (здесь и~далее для
краткости опускаем волну над~$S$)
	\[
	\frac{\partial S}{\partial t}+H\left(t,\frac{\partial S}{\partial q},q\right)=0.
	\eqtag{eq:hamiltonian-jacobi}
	\]
Это уравнение \emph{Гамильтона~— Якоби.} Так называют дифференциальные
уравнения с~частными производными (или обыкновенные) первого порядка, не
содержащие зависимую переменную. Если~бы удалось найти решение $S(t,q,\alpha)$
этого уравнения, содержащее набор параметров
$\alpha=(\alpha_1,\ldots,\alpha_n)$, можно было~бы воспользоваться этим
решением как производящей функцией канонического преобразования от переменных
$(p,q)$ к~переменным $(\alpha,\beta)$, в~которых преобразованные канонические
уравнения приобретают тривиальный вид
	\[
	\dot\alpha=\dot\beta=0
	\]
и
	\[
	p=\frac{\partial S}{\partial q},
	\quad
	c\beta=\frac{\partial S}{\partial\alpha}.
	\eqtag{eq:trans-formulae}
	\]
Для разрешимости последних уравнений относительно $(p,q)$ потребуем
дополнительно
	\[
	\operatorname{rank}\left(\frac{\partial^2S}{\partial\alpha_i\partial q_j}\right)=n,
	\eqtag{eq:nondeg}
	\]
а~для простоты возьмём $c=1$. Семейство решений \eqref{eq:hamiltonian-jacobi},
параметризованное параметрами $\alpha$ с~выполнением \eqref{eq:nondeg}
называется \emph{полным интегралом\/} уравнения Гамильтона~— Якоби. Не нужно
его путать с~общим интегралом (решением)~— общее решение уравнений с частными
производными в~типичном случае образует бесконечномерное семейство; нам~же
достаточно иметь $n$"=мерное.

В~новых фазовых переменных $(\alpha,\beta)$ устанавливается полный покой (все
они являются первыми интегралами), а~это означает, что каноническое
преобразование с~полным интегралом уравнения Гамильтона~— Якоби, взятым
в~качестве производящей функции, совпадает с~фазовым потоком исходных
канонических уравнений (\emph{фазовым потоком системы обыкновенных
дифференциальных уравнений} называется отображение, переводящее начальную точку
фазового пространства в~решение уравнений в~момент времени~$t$). Это полностью
аналогично переходу к~системе координат, координатные линии которого увлекаются
потоком некоторой воображаемой фазовой жидкости~— в~таких координатах все
частицы жидкости, очевидно, покоятся.

Осталось лишь разрешить уравнения \eqref{eq:trans-formulae} относительно
переменных $(p,q)$:
	\[
	p=p(t,\alpha,\beta),
	\quad
	q=q(t,\alpha,\beta).
	\]
Решение канонических уравнений получено методом Якоби.

Таким образом, знание полного интеграла уравнения Гамильтона~— Якоби есть
безусловное благо. Но как его найти? Вообще говоря, это трудное и не всегда
возможное дело. Иногда полный интеграл получается подобрать. В~некоторых
чрезвычайно редких, но чрезвычайно важных в~теоретической механике случаях
удаётся найти его \emph{методом разделения переменных.}

\section{Метод разделения переменных}

Самое общее уравнение Гамильтона~— Якоби имеет вид
	\[
	\Phi\left(x,\frac{\partial S}{\partial x}\right)=0.
	\eqtag{eq:hamiltonian-jacobi-generic}
	\]
В~нашем случае $x=(t,q)$.

Предположим, что $x=(x_1,x')$, причём
	\[
	\Phi\left(x,\frac{\partial S}{\partial x}\right)
		=\tilde\Phi\left(\varphi\left(x_1,\frac{\partial S}{\partial x_1}\right),x',\frac{\partial S}{\partial x'}\right).
	\eqtag{eq:separation}
	\]
В~таком случае говорят, что переменная $x_1$ \emph{отделяется\/} от
остальных переменных~$x'$ (штрих здесь просто значок). Тогда решение уравнения
\eqref{eq:separation} можно искать в~виде
	\[
	S(x)=S_1(x_1)+S'(x'),
	\]
и~при подстановке в~\eqref{eq:hamiltonian-jacobi-generic} получится
	\[
	\tilde\Phi\left(\varphi_1\left(x_1,\frac{\partial S_1(x_1)}{\partial x_1}\right),
		x',\frac{\partial S'(x')}{\partial x'}\right)=0.
	\eqtag{eq:separation-1step}
	\]
Это равенство должно удовлетворяться тождественно при любых $(x_1,x')$. Если
изменять лишь переменную~$x_1$, меняться в~этом равенстве может лишь выражение
$\varphi_1(\ldots)$. С учётом этого получим, что
$\varphi_1(\ldots)=\kappa_1=\mathup{const}$, иначе тождественного обращения левой
части в~ноль \emph{(обнуления)} не получится.

Если, к~примеру, $x_1=q_s$, то, таким образом,
$\varphi_1(q_s,\slashfrac{\partial S_1}{\partial
q_s})=\varphi_1(q_s,p_s)=\kappa_1$~— первый интеграл. Если~же $x_1=t$, то
$\varphi_1(t,\slashfrac{\partial S_1}{\partial
t})=\varphi_1(t,H(t,p,q))=\kappa_1$~— опять первый интеграл. Отделившаяся
переменная всегда даёт первый интеграл.

Уравнение
	\[
	\varphi_1\left(x_1,\frac{\partial S_1}{\partial x_1}\right)=\kappa_1
	\]
в сущности является обыкновенным. Разрешив его относительно производной,
получим
	\[
	\frac{\partial S_1}{\partial x_1}=\psi_1(x_1,\kappa_1),
	\quad
	S_1(x_1)=\int\psi_1(x_1,\kappa_1)\,dx_1.
	\]

Из \eqref{eq:separation-1step} получим
	\[
	\tilde\Phi\left(\kappa_1,x',\frac{\partial S'(x')}{\partial x'}\right)=0.
	\]
Это снова уравнение Гамильтона~— Якоби, но теперь относительно функции $S'(x')$
с~меньшим числом аргументов. Попытаемся применить к~нему тот~же приём.

Если повезёт, рекурсивное применение процедуры отделения переменной приведёт
к~полному разделению всех переменных, и~будет найдено параметрическое семейство
решений уравнения Гамильтона~— Якоби
	\[
	S(x,\kappa)=\sum_{i=1}^nS_i(x_i,\kappa_1,\ldots,\kappa_i)
	\]
с~нужным числом параметров. Попутно будут найдены $n$~первых интегралов.

Заметим, что случаи $\slashfrac{\partial H}{\partial q_s}=0$
и~$\slashfrac{\partial H}{\partial t}=0$ приводят
к~$\varphi_1=p_s=\mathup{const}$ и~$\varphi_1=H(p,q)=\mathup{const}$
соответственно. Это уже известные нам циклический интеграл и~интеграл энергии
(Якоби).

В~автономном случае ($\slashfrac{\partial H}{\partial t}=0$) удобно сразу
положить
	\[
	S=-\kappa_1t+S'(q),
	\]
и~искать полный интеграл \emph{укороченного} уравнения Гамильтона~— Якоби
	\[
	H\left(q,\frac{\partial S'}{\partial q}\right)=\kappa_1
	\]
любым доступным способом, хоть~бы и~разделением переменных.

Метод Якоби является очень мощным, несмотря на узость круга задач, где он
применим. Эти методом проинтегрированы те лагранжевы задачи, где другие методы
не сработали.

Один из примеров: движение по инерции по квадрике в трёхмерном пространстве.
Вводятся специальные координаты, в которых метрика на квадрике приводится
к~лиувиллеву виду (см.~задачу~\ref{prb:2}), после чего переменные разделяются,
и~находится ещё один (помимо очевидного интеграла энергии) первый интеграл,
квадратичный по отношению к скоростям. Как интересное следствие, интегрируемой
оказывается задача о~биллиарде, ограниченном квадрикой на плоскости. Устремив
к~нулю одну из осей квадрики в~трёхмерном пространстве, получим двулистно
накрытую плоскую фигуру, ограниченную кривой второго порядка. Траектории
движения по инерции на пространственной квадрике перейдут в биллиардные. Первый
интеграл при предельном переходе превращается в~дополнительный первый интеграл
биллиардной задачи. Двух первых интегралов (энергии и дополнительного)
оказывается достаточно для нахождения биллиардного движения при помощи
квадратур.

Другой знаменитый пример успешного применения метода Якоби~—
задача~\ref{prb:3}.

\section{Задачи}

\begin{enumerate}

\item
Найдите производящую функцию тождественного канонического преобразования
\emph{другого типа.}

\item
Рассмотрим величину
	\[
	S(t,q)=\int_{t^*}^tL(\theta,q(\theta),\dot q(\theta))\,d\theta,
	\]
где $q(t)$ суть решения уравнений Эйлера~— Лагранжа $\mbfsansE(L)=0$ такие, что
$q(t^*)=q^*$, $q(t)=q$. Докажите, что функция $S(t,q)$ удовлетворяет уравнению
Гамильтона~— Якоби \eqref{eq:hamiltonian-jacobi}, в~котором гамильтониан~$H$
отвечает лагранжиану~$L$. Величина $S$ называется \emph{действием.}

\item\label{prb:2}
А. Метрика на двумерной поверхности в~некоторых координатах $(x,y)$ имеет вид
$T=(u(x)+v(y))(\dot x^2+\dot y^2)$ (метрики такого вида называют
\emph{лиувиллевыми}). Найдите движения по инерции по таким поверхностям методом
Якоби. Выпишите два независимых первых интеграла.

Б.~Поверхность вращения задана в~цилиндрических координатах $(r,\theta,z)$
уравнением $r=f(z)$ (получена вращением графика функции $f(z)$ вокруг оси~$z$).
Покажите, что в~некоторых координатах метрика на поверхности вращения может
быть представлена в~лиувиллевом виде. \emph{Подсказка:\/}~одна из
координат~— $\theta$, а~вторая~— $\zeta=\zeta(z)$, найдите её.

\item\label{prb:3}
4.27.

\item
Прямая вращается в~вертикальной плоскости вокруг некоторой своей точки
с~угловой скоростью~$1$ (в~нулевой момент времени прямая горизонтальна). По
прямой движется материальная точка массы~$1$ без трения под действием силы
тяжести ($g=1$). Обобщённая координата $x$~— расстояние от точки до центра
вращения. Найдите движение методом Якоби. \emph{Подсказка:\/}~Попытайтесь найти
полный интеграл уравнения Гамильтона~— Якоби разделением переменных. Когда это
не удастся сделать, постарайтесь найти его подбором. Полный интеграл ищите
в~виде $S(t,x)=A(t)x^2+B(t)x+C(t)$. Превратите уравнение Гамильтона~— Якоби
в~систему обыкновенных уравнений относительно $A$, $B$, $C$. Не старайтесь
найти самое общее решение полученной системы~— упрощайте. Помните, что полный
интеграл должен содержать лишь один параметр (общее решение системы содержит
три). Не забывайте про условие невырожденности. Сравните полученное методом
Якоби решение с~решением, полученным непосредственным интегрированием
лагранжевых уравнений. Должно совпасть!

%\item
%\textsc{[To be continued]}

\end{enumerate}

Будьте здоровы!

\end{document}
