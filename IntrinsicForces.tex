\documentclass[a4paper,11pt,draft]{article}
\usepackage{luatex85}
\usepackage[russian]{hyperref}
\usepackage[babelshorthands]{polyglossia}
\usepackage[fleqn,reqno]{amsmath}
\usepackage{amsthm}
%\usepackage[math-style=ISO,bold-style=upright,vargreek-shape=unicode,partial=italic]{unicode-math}
\usepackage[math-style=ISO,bold-style=upright,partial=italic]{unicode-math}
\usepackage{microtype}
\usepackage[width=14.85cm]{geometry}

\setmainlanguage{russian}
\setmainfont{Cambria}
\setmathfont{Cambria Math}[sans-style=literal]
\setmathfont{XITS Math}[range={\mathscr,\perp,\leqslant,\geqslant}]

\makeatletter

\newcommand\eqtag[1]{\refstepcounter{equation}\tag{\theequation}\label{#1}}

%\newcommand\slashfrac[2]{{#1\fracslash#2}}
\newcommand\slashfrac[2]{{#1/#2}}

\makeatother

\theoremstyle{definition}
\newtheorem{theorem}{Теорема}
\newtheorem*{theorem*}{Теорема}
\newtheorem{lem}{Лемма}
\newtheorem*{lem*}{Лемма}
\newtheorem{example}{Пример}

\title{Силы реакций идеальных связей в~твёрдом теле являются внутренними}
\author{А.~Н.~Швец}

\AtBeginDocument{\hypersetup{pdftitle=\@title}}{}

\begin{document}

\maketitle

Твёрдым телом назовём систему материальных точек, попарные расстояния между
которыми постоянны. Твёрдое тело можно рассматривать как систему с~голономными
связями с~уравнениями
	\[
	(\mbfr_\alpha-\mbfr_\beta)^2-c_{\alpha\beta}^2=0,
	\quad
	\alpha,\beta=1,\ldots,N,
	\eqtag{eq:1}
	\]
где $\mbfr_\nu$~— радиусы"=векторы точек, $N$~— количество точек,
$c_{\alpha\beta}$~— константы.

Согласно принципу освобождения от связей твёрдое тело можно также рассматривать
как систему свободных материальных точек под действием дополнительных (помимо
заданных) сил реакции $\mbfR_\nu$.

В~динамике к~определению твёрдого тела добавляют требование, чтобы силы реакции
были внутренними, то~есть допускали представление в~виде
	\[
	\mbfR_\nu=\sum_\kappa\lambda_{\nu\kappa}(\mbfr_\nu-\mbfr_\kappa),
	\quad
	\lambda_{\nu\kappa}=\lambda_{\kappa\nu},
	\eqtag{eq:2}
	\]
где $\lambda_{\nu\kappa}$~— некоторые константы.

Нетрудно доказать, что с~этим дополнительным требованием связи в твёрдом теле
являются идеальными. В~самом деле, с~учётом условий на виртуальные перемещения
	\[
	\langle\mbfr_\alpha-\mbfr_\beta,\delta\mbfr_\alpha-\delta\mbfr_\beta\rangle=0,
	\quad
	\alpha,\beta=1,\ldots,N
	\eqtag{eq:3}
	\]
форма виртуальной работы сил реакции равна нулю:
	\[
	\sum_\nu\langle\mbfR_\nu,\delta\mbfr_\nu\rangle
		=\sum_\nu\sum_\kappa\lambda_{\nu\kappa}
		\langle\mbfr_\nu-\mbfr_\kappa,\delta\mbfr_\nu\rangle=0
	\]
(последняя сумма состоит из пар слагаемых вида
	\[
	\lambda_{\nu\kappa}\langle\mbfr_\nu-\mbfr_\kappa,\delta\mbfr_\nu\rangle
		+\lambda_{\kappa\nu}\langle\mbfr_\kappa-\mbfr_\nu,\delta\mbfr_\kappa\rangle,
	\]
которые уничтожаются парами в~силу~\eqref{eq:3} и симметричности матрицы
$\lambda_{\nu\kappa}$).

Верно и обратное утверждение.

\begin{theorem*}
Пусть связи в~твёрдом теле идеальны. Тогда силы их реакции являются
внутренними.
\end{theorem*}

\begin{lem*}
Пусть $f,g_s\colon\BbbR^n\to\BbbR$~— линейные функции, причём $f(x)=0$ в~силу
$g_s(x)=0$, $s=1,\ldots,k$, $x\in\BbbR^n$. Тогда найдутся множители
$\lambda_s\in\BbbR$ такие, что $f(x)\equiv\sum_{s=1}^k\lambda_s g_s(x)$.
\end{lem*}

\begin{proof}[Доказательство леммы]
Положим $V_s=\ker g_s$. Тогда по условию $f\in(\bigcap_sV_s)^\perp$ (здесь
$V^\perp$ означает ортогональное дополнение к~подпространству~$V$, то~есть
$\{h\in(\BbbR^n)^*\mid V\subset\ker h\}$). Утверждение леммы равносильно
равенству
	\[
	\left(\bigcap_sV_s\right)^\perp=\sum_sV_s^\perp.
	\]
Заменив в~этом равенстве $V_s$ на $V_s^\perp$ и взяв ортогональное дополнение
к~обеим частям, получим равносильное равенство
	\[
	\bigcap_sV_s^\perp=\left(\sum_sV_s\right)^\perp.
	\]
Его справедливость вытекает из того факта, что
принадлежность линейной функции~$h$ пространствам в~обеих частях равенства
равносильна принадлежности $h$ каждому из пространств $V_s^\perp$,
$s=1,\ldots,k$. Лемма доказана.
%
%Пусть $h\in\left(\sum_sV_s\right)^\perp$, тогда $h\in V_s^\perp$ при
%$s=1,\ldots,k$, следовательно, $h\in\bigcap_sV_s^\perp$. Обратно, пусть
%$h\in\bigcap_sV_s^\perp$, тогда $h\in V_s^\perp$ при $s=1,\ldots,k$,
%следовательно, $h\in\left(\sum_sV_s\right)^\perp$. Лемма доказана.
\end{proof}

\begin{proof}[Доказательство теоремы]
Идеальность связей в~твёрдом теле означает равенство нулю формы виртуальной
работы сил реакции, то есть обращение в~ноль линейной функции
$\sum_\nu\langle\mbfR_\nu,\delta\mbfr_\nu\rangle$ на любом векторе
$(\delta\mbfr_1,\ldots,\delta\mbfr_N)$, удовлетворяющем условиям~\eqref{eq:3},
также линейным по отношению к~виртуальным перемещениям.

Согласно лемме найдутся такие множители $\lambda_{\alpha\beta}$, что будет
выполняться равенство
	\[
	\sum_\nu\langle\mbfR_\nu,\mbfv_\nu\rangle
		=\sum_{\alpha,\beta}\lambda_{\alpha\beta}
		\langle\mbfr_\alpha-\mbfr_\beta,\mbfv_\alpha-\mbfv_\beta\rangle
	\eqtag{eq:4}
	\]
для всевозможных векторов $(\mbfv_1,\ldots,\mbfv_N)$, не стеснённых никакими
ограничениями.

Расщепляя уравнение~\eqref{eq:4}, получим равенства
	\[
	\mbfR_\nu=\sum_\kappa[\lambda_{\nu\kappa}(\mbfr_\nu-\mbfr_\kappa)
		-\lambda_{\kappa\nu}(\mbfr_\kappa-\mbfr_\nu)]
		=\sum_\kappa(\lambda_{\nu\kappa}+\lambda_{\kappa\nu})(\mbfr_\nu-\mbfr_\kappa),
	\]
то~есть силы реакций связей имеют вид~\eqref{eq:2}, и, следовательно, являются
внутренними. Теорема доказана.
\end{proof}

\end{document}
