\documentclass[a4paper,11pt]{article}
\usepackage{luatex85}
\usepackage[babelshorthands]{polyglossia}
\usepackage[fleqn,reqno]{amsmath}
\usepackage{amsthm}
\RequirePackage[
	backend=biber,
	bibstyle=gost-numeric,
	citestyle=gost-numeric,
	%defernumbers=true,
	defernumbers=false,
	language=auto,
	autolang=langname,
	]{biblatex}
\usepackage[math-style=ISO,bold-style=upright,partial=italic]{unicode-math}
\usepackage{microtype}
\usepackage[width=16cm,height=24cm]{geometry}
\usepackage[russian]{hyperref}

\setmainlanguage{russian}

\setmainfont{Cambria}
\setsansfont{Calibri}
\setmonofont{Source Code Pro}[Scale=MatchLowercase]
\setmonofont{Source Code Pro}[Scale=MatchLowercase]
\setmathfont{Cambria Math}[sans-style=literal]
\setmathfont{XITS Math}[range={\mathscr}]

\addbibresource{\jobname.bib}

\makeatletter

\allowdisplaybreaks[4]

\def\[#1\]{\begin{align*}#1\end{align*}}
\newcommand\eqtag[1]{\refstepcounter{equation}\tag{\theequation}\label{#1}}


\title{Дифференциал операции продолжения группы}
\author{В. Боровик}
\date{3 мая 2020}

 \newcommand{\pr}{\mathbf{pr}}
 \newcommand{\vv}{\mathbf{v}}
 \newcommand{\ww}{\mathbf{w}}

\begin{document}
\maketitle
\sloppy

Пусть $M$~—  гладкое многообразие с локальными координатами $x =
(x_1,\ldots,x_n)$, на котором действует группа  преобразований $G=\{g^\varepsilon\}$ по правилу: $g^\varepsilon\cdot x=\hat x$.
Определим операцию \textit{первого продолжения\/} группы $G$:
	\[
	\pr^{(1)}\colon\{g^{\varepsilon}\} \rightarrow \{\left(g^{\varepsilon},(g^{\varepsilon})^*\right)\},
	\]
где $(g^{\varepsilon})^*$~— дифференциал преобразования $g^\varepsilon$.

Группа $\pr^{(1)}G = \{\left(g^{\varepsilon},(g^{\varepsilon})^* \right)\}$ действует уже на касательном расслоении $M^* = \{(x, \dot x)\}$ по правилу:
	\[
	(g^{\varepsilon},(g^{\varepsilon})^* ) \cdot (x, \dot x) = (\hat x, \hat{\dot x}),
	\]
где $\hat{\dot x}$ определяется условиями согласованности Пфаффа
	\[
	(d\hat x_i=\hat{\dot x}\,dt)\bmod(d x_j = \dot x_j\,dt),
	\quad
	i,j=1,\ldots,n.
	\]
Отображение $\pr^{(1)}$ осуществляет гомоморфизм локальных групп Ли в~силу
следущих соотношений:
	\[
	\pr^{(1)}g^{\alpha+\beta}=\left(g^{\alpha+\beta},(g^{\alpha+\beta})^*\right)
		=(g^\beta\circ g^\alpha,(g^\beta)^*\circ(g^\alpha)^*)
		=\pr^{(1)}g^\alpha\circ\pr^{(1)}g^\beta,
	\]
здесь мы пользуемся тем, что дифференциал композиции отображений является
композицией дифференциалов.

Раннее мы установили соответствие между однопараметрическими группами
преобразований $M$ и векторными полями на $M$. 
	\[
	g^\varepsilon=\mathbf{exp}(\varepsilon \vv),
	\quad
	\vv=\frac d{d\varepsilon}\bigg |_{\varepsilon = 0}g^{\varepsilon}.
	\]
Если рассматривать $g^\varepsilon\colon\mathbb{R}\to G$ как кривую на
многообразии $G$, проходящую через единицу, то из определения $\vv$ следует,
что это касательный вектор в смысле классов эквивалентности соприкасающихся
кривых. Таким образом, $\vv\in T_{g^0}G$~—  касательному пространству в~единице
группы Ли~$G$, которое, как известно, отождествляется с~алгеброй Ли группы~$G$,
в~дальнейшем будем обозначать ее $\mathcal{Lie}(G)$. Аналогично, элементами
алгебры Ли $\mathcal{Lie}(\pr^{(1)}G)$ группы $\pr^{(1)}G$ являются генераторы
$\pr^{(1)}\vv$:
	\[
	\pr^{(1)}\vv=\frac d{d \varepsilon}\bigg|_{\varepsilon = 0}\pr^{(1)}g^{\varepsilon}.
	\]
С~другой стороны, дифференциал описанного отображения $d\pr^{(1)}$ сопоставляет
касательному вектору $\vv\in\mathcal{Lie}(G)$ касательный вектор
$d\pr^{(1)}(\vv) \in \mathcal{Lie}(\pr^{(1)}G)$ по правилу:
	\[
	d\pr^{(1)}(\vv)=d\pr^{(1)}\left(\frac d{d\varepsilon}\bigg|_{\varepsilon=0}g^\varepsilon\right)
		=\frac{d}{d\varepsilon}\bigg |_{\varepsilon=0} \left( \pr^{(1)}\circ g^\varepsilon\right)
		=\frac{d}{d\varepsilon}\bigg |_{\varepsilon=0}\pr^{(1)}g^\varepsilon=\pr^{(1)}\vv.
	\]
Так как мы определили касательный вектор, как класс соприкасающихся кривых, то
использовали соответствующее определение дифференциала:
	\[
	d_x\phi(\gamma'(0))=(\phi \circ \gamma)'(0),
	\]
где $\phi\colon M\to N$, $\gamma$~— кривая в~$M$, такая что $\gamma(0)=x$.

Таким образом, получается, что операция продолжения векторных полей является
дифференциалом отображения $\pr^{(1)}$ между $G$ и $\pr^{(1)}G$.
	\[
	&\pr^{(1)}: G \rightarrow \pr^{(1)}G,\\
	&d\pr^{(1)}: \mathcal{Lie}(G) \rightarrow \mathcal{Lie}(\pr^{(1)}G).
	\]
Дифференциал гомоморфизма групп Ли является гомоморфизмом алгебр Ли, в силу
этого получаем соотношение:
	\[
	\pr^{(1)}[\vv,\ww]=[\pr^{(1)}\vv, \pr^{(1)}\ww],
	\quad
	\forall \vv,\ww\in\mathcal{Lie}(G).
	\]

\end{document}
