\documentclass[a4paper,11pt]{article}
\usepackage{luatex85}
\usepackage{comment}
\usepackage[warn]{mathtext}
%\usepackage[T1,T2A]{fontenc}
%\usepackage[utf8]{inputenc}
\usepackage[babelshorthands]{polyglossia}
\usepackage[babelshorthands]{polyglossia}
%\usepackage[russian, english]{babel}
%\usepackage{indentfirst}
\usepackage{amsmath}
\usepackage{amscd}
\usepackage{amssymb}
\usepackage{amsfonts}
%\usepackage{euscript}
%\usepackage{geometry} \geometry{verbose,a4paper,tmargin=1.5cm,bmargin=4cm,lmargin=2.5cm,rmargin=2.5cm}
\usepackage[width=16cm,height=24cm]{geometry}
\usepackage{unicode-math}
\usepackage{microtype}
%\renewcommand{\baselinestretch}{1}
%\linespread{1.4}
%\setlength{\parskip}{0.15cm}
\def\[#1\]{\begin{align*}#1\end{align*}}

\setmainlanguage{russian}

\setmainfont{Cambria}
\setsansfont{Calibri}
\setmonofont{Source Code Pro}[Scale=MatchLowercase]
\setmonofont{Source Code Pro}[Scale=MatchLowercase]
\setmathfont{Cambria Math}[sans-style=literal]
\setmathfont{XITS Math}[range={\mathscr}]


\begin{document}
\author{В.~Белослудцев}
\title{Продолжения векторных полей}
\maketitle

\section{Вывод общей формулы}

Для того, чтобы в~голове не возникало каши из дифференциальной геометрии
и~аналитической механики, проведём небольшое повторение: 

В каждый момент времени для точки определена координата $x$, и~её скорость
$\dot x$. Вместе со~временем~$t$ эти три параметра дают нам расширенное
фазовое пространство $\tilde M\ni(t,x,\dot x)$~— всевозможные варианты движения
точки. Для простоты мы считаем, что координаты~$x$ точки образуют гладкое
многообразие $M$. А раз многообразие гладкое, то можем говорить и~о~касательном
расслоении $M^*\ni(x,\dot x)$.

\vspace{0.6 cm}

\textbf{Резюме:}

В~\emph{конфигурационном пространстве} живут координаты точек $x$.

В~\emph{расширенном конфигурационном пространстве} живут координаты точек $x$
и~время $t$.

В \emph{расширенном фазовом пространстве} живут координаты точек $x$, скорости
$\dot x$ и~время~$t$.

\vspace{0.6 cm}

Ввиду приведённых выше соображений из механики, буквы $(t,x,\dot x)$ не
являются полностью независимыми. Между ними имеется дифференциальное
соотношение $\frac{dx}{dt}=\dot x$. Нужно, чтобы и~после преобразований
фазовым потоком (речь о нём пойдёт ниже) это соотношение сохранилось. Сие
и~выражено в~условиях Пфаффа.

Предположим, что теперь из всевозможных движений точки мы хотим выбрать
конкретные. Для этого будем рассматривать дифференциальные уравнения на
расширенном фазовом пространстве. Эти уравнения соответствуют какому"=то
фазовому потоку (обозначим соответствующую однопараметрическую группу через
$G$), который в свою очередь даст нам обобщённое векторное поле
$\mathbf{v}=\sum_{j=1}^n\xi_{j}\partial_{x_{j}}+\tau\partial_t$. Здесь нет
дифференцирования по $\dot{x}$, так~как пока мы считаем, что группа действует
только на множестве координат и~на времени. А~теперь рассмотрим действие этой
группы на всём расширенном фазовом пространстве. Действие на $x$ и $t$ мы уже
знаем. В~листке показывалось, как через дифференциал доопределить действие на
$\dot{x}$. Теперь~же мы хотим понять, как будут выглядеть генераторы нашего
нового действия $\mathbf{pr^{(1)}}G$:
	\[
	g^{\varepsilon}(x)\colon\tilde M\to\tilde M,
	\quad
	(t,x,\dot x)\mapsto(\hat x,\hat{\dot x},\hat t).
	\]

Коэффициенты генератора для $\partial_{x}$ и $\partial_{t}$ мы уже знаем.
Осталось найти коэффициенты при $\partial_{\dot{x}}$. Обозначим их через
$\eta=\frac{d\hat{\dot x}}{d\varepsilon}\Bigr|_{\varepsilon=0}$.

Условия Пфаффа гласят: если $dx_{j}=\dot{x}_{j}\,dt$, то $d\hat
x_j=\hat{\dot{x}}_{j}\,d\hat{t}$. 

Разложим слагаемые из последнего равенства в ряд по $\varepsilon$ до первой
степени:
	\[
	&\hat x_j=x_j+\varepsilon\xi_j+o(\varepsilon),\\
	&\hat{\dot{x}}_j=\dot x_j+\varepsilon\eta_j+o(\varepsilon),\\
	&\hat{t}=t+\varepsilon\tau+o(\varepsilon).
	\]

Подставив эти разложения в~условия Пфаффа, получим следующее (что-то успешно
сократится):
	\[
	d(x_j+\varepsilon\xi_j+o(\varepsilon))
		-(\dot x_j+\varepsilon\eta_j+o(\varepsilon))\,d(t+\varepsilon\tau+o(\varepsilon))=0.
	\]
Следовательно,
	\[
	\varepsilon\eta_{j}\,dt+o(\varepsilon)
		=\varepsilon\,d\xi_{j}-\varepsilon\dot{x}_{j}\,d\tau+o(\varepsilon).
	\]
Значит,
	\[
	\frac{d\hat{\dot x}_j}{d\varepsilon}\Biggr|_{\varepsilon=0}=\eta_j=\dot\xi_{j}-\dot\tau\dot x_j
		=(\xi_j-\tau\dot x_j)\dot{\,}+\tau\ddot x_j.
	\]
Таким образом, новое векторное поле выглядит так\footnote{Заметим, что
полученная формула в~частном случае, когда $\tau=0$, даёт формулу
из методички (да и~могло~ли быть иначе?).~\emph{— А.~Ш.}}:
	\[
	\mathbf{pr^{(1)}v}=\mathbf{v}+\sum_{j=1}^n\eta_j\partial_{\dot x_j}
		=\sum_{j=1}^n\xi_j\partial_{x_j}+\tau\partial_t
		+\sum_{j=1}^n({\dot\xi}_j-\dot\tau\dot x_j)\partial_{\dot x_j}.
	\]

\end{document}
