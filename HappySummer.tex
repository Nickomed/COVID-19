\documentclass[a4paper,11pt]{article}
\usepackage{luatex85}
\usepackage[babelshorthands]{polyglossia}
\usepackage[fleqn,reqno]{amsmath}
\usepackage{amsthm}
\usepackage[math-style=ISO,bold-style=upright,partial=italic]{unicode-math}
\usepackage{microtype}
\usepackage[width=16cm,height=24cm]{geometry}
\usepackage[russian]{hyperref}
\usepackage{luamplib}

\setmainlanguage{russian}

\setmainfont{Cambria}
\setsansfont{Calibri}
\setmonofont{Source Code Pro}[Scale=MatchLowercase]
\setmonofont{Source Code Pro}[Scale=MatchLowercase]
\setmathfont{Cambria Math}[sans-style=literal]
\setmathfont{XITS Math}[range={\mathscr}]

\makeatletter

\allowdisplaybreaks[4]

\def\[#1\]{\begin{align*}#1\end{align*}}
\newcommand\eqtag[1]{\refstepcounter{equation}\tag{\theequation}\label{#1}}

%\newcommand\slashfrac[2]{{#1\fracslash#2}}
\newcommand\slashfrac[2]{{#1/#2}}

\newcommand\pr{\operatorname{\textbf{\textup{pr}}}}

\theoremstyle{definition}
\newtheorem{theorem}{Теорема}
\newtheorem*{theorem*}{Теорема}
\newtheorem{lemma}{Лемма}
\newtheorem{definition}{Определение}
\newtheorem{example}{Пример}
%\newtheorem{proof}{Доказательство}

\newcommand\metasetup{\hypersetup{
	pdftitle=\@title,
	pdfauthor=\@author,
	linkbordercolor={0 .5 .25},
	}}

\setlength\overfullrule{5pt}

\makeatother

\begin{document}

\hyphenation{Ла-гран-жа ла-гран-же-вой}

\title{Домашнее задание на~лето}
\author{А.~Н.~Швец}

\metasetup
\maketitle

На множестве четырёхугольников на плоскости рассмотрим преобразование,
переводящее четырёхугольник $\mbfx_1\mbfx_2\mbfx_3\mbfx_4$
в~$\mbfy_1\mbfy_2\mbfy_3\mbfy_4$, где $\mbfy_1,\mbfy_2,\mbfy_3,\mbfy_4$ суть
ортоцентры треугольников $\mbfx_2\mbfx_3\mbfx_4$, $\mbfx_1\mbfx_3\mbfx_4$,
$\mbfx_1\mbfx_2\mbfx_4$, $\mbfx_1\mbfx_2\mbfx_3$ соответственно
(рис.~\ref{fig:1}).

\begin{figure}[h]
\centering
\leavevmode
\begin{mplibcode}
input elementi;
input colorbrewer-rgb;

beginfig(0)
u:=cm;

z1=origin;
z2=(6u, -.5u);
z3=(5u, 3u);
z4=(2u, 4u);
z5=orthocenter(z1, z2, z3);
z6=orthocenter(z2, z3, z4);
z7=orthocenter(z3, z4, z1);
z8=orthocenter(z4, z1, z2);

draw altitude(z1, z2, z3) withcolor Greys[3][2];
draw z1--z7 withcolor Greys[3][2];
draw z2--z6 withcolor Greys[3][2];
draw altitude(z2, z4, z1) withcolor Greys[3][2];
draw z3--z7 withcolor Greys[3][2];
draw altitude(z3, z1, z2) withcolor Greys[3][2];
draw altitude(z4, z1, z2) withcolor Greys[3][2];
draw z4--z6 withcolor Greys[3][2];

draw z1--z2--z3--z4--cycle withpen pencircle scaled bp withcolor Blues[3][3];
draw z5--z6--z7--z8--cycle withpen pencircle scaled bp withcolor Greens[3][3];

label.bot(textext("$\mbfx_1$"), z1);
label.bot(textext("$\mbfx_2$"), z2);
label.top(textext("$\mbfx_3$"), z3);
label.top(textext("$\mbfx_4$"), z4);

label.bot(textext("$\mbfy_4$"), z5);
label.top(textext("$\mbfy_1$"), z6);
label.top(textext("$\mbfy_2$"), z7);
label.bot(textext("$\mbfy_3$"), z8);

endfig
\end{mplibcode}
\caption{Построение равновеликого четырёхугольника}\label{fig:1}
\end{figure}

Посмотреть на преобразование в~действии
\href{https://www.geogebra.org/m/uadvg7bz}{можно} на сайте Геогебры.

\begin{enumerate}
\item
Проверьте, верно~ли, что преобразование является каноническим относительно
симплектической 2"=формы в~8"=мерном пространстве координат вершин
четырёхугольника.
	\[
	\omega=\sum_{i=1}^4dx_i^1\wedge dx_i^2,
	\]
где $(x_i^1,x_i^2)$~— координаты вершины $\mbfx_i$.

\item
Докажите, что преобразование сохраняет площадь
четырёхугольника\footnote{Об~описанном отображении и~данном его свойстве нам
в~1998~г. в~частном разговоре сообщил Д.~В.~Трещёв со~ссылкой,
предположительно, на В.~Е.~Подольского.}, иными словами, площадь является
первым интегралом отображения.

\item
Докажите, что все 8~вершин четырёхугольников $\mbfx_1\mbfx_2\mbfx_3\mbfx_4$
и~$\mbfy_1\mbfy_2\mbfy_3\mbfy_4$ лежат на некоторой квадрике.

\item
Пользуясь ранее доказанным, докажите, что названная квадрика есть равнобочная
гипербола.

\item
Пользуясь ранее доказанным, докажите, что отображение обладает четырьмя
независимыми первыми интегралами, рациональными относительно координат вершин
(осторожно, очень громоздкие выражения, пользуйтесь CAS).

\item
Выразите площадь как функцию найденных первых интегралов.

\item
Составьте таблицу скобок Пуассона найденных первых интегралов.

\end{enumerate}

\end{document}
