\documentclass[11pt,a4paper]{article}
\usepackage{amsmath}
\usepackage{amssymb}
\usepackage{amsthm}
\usepackage{enumerate}
%\usepackage{latexsym}
%\usepackage{russcorr}
% \usepackage[cp1251]{inputenc}
%\usepackage[utf8]{inputenc}
\usepackage[babelshorthands]{polyglossia}
\usepackage{unicode-math}
\usepackage{calrsfs}
%\usepackage{dsfont}
%\usepackage{cancel}
\usepackage[russian]{hyperref}
\usepackage{microtype}
\usepackage[width=16cm,height=24cm]{geometry}

\setdefaultlanguage{russian}
\setmainfont{Cambria}
\setmathfont{Cambria Math}

\newtheorem{Theorem}{Теорема}[section]
\newtheorem{Corollary}[Theorem]{Следствие}
\newtheorem{Lemma}[Theorem]{Лемма}
\newtheorem{Example}[Theorem]{Пример}
\newtheorem{Conjecture}[Theorem]{Гипотеза}
\newtheorem{Proposition}[Theorem]{Предложение}
\theoremstyle{definition}
\newtheorem{Definition}[Theorem]{Определение}
\newtheorem{Notation}[Theorem]{Обозначение}
\newtheorem{Question}[Theorem]{Вопрос}
\newtheorem{Remark}[Theorem]{Замечание}
\renewcommand{\proofname}{Доказательство.}
\numberwithin{equation}{section}



\def\e{{\varepsilon}}
\def\R{{\mathbb R}}

\def\L{{\cal L}}


\begin{document}
\author{Н.~Колегов}
\title{Ядро оператора Эйлера}
\maketitle
\section{Оператор Эйлера}
Пусть $L(t,q(t),\dot{q}(t))$~— функция Лагранжа, где $q(t)=(q_1(t),\ldots,
q_m(t))$. Будем считать далее, что функции $L,q$ достаточно гладкие, а также
$t\in\Omega\subseteq\R$, где $\Omega$~— область на прямой. Определим {\em
оператор Эйлера} $\mbfsansE$ как набор дифференциальных операторов $\mbfsansE=(\mbfsansE_1,\ldots,
\mbfsansE_m)$, действующих по следующему правилу
$$\mbfsansE_\alpha(L)=\dfrac{\partial L}{\partial q_\alpha}-\dfrac{d}{dt}\dfrac{\partial L}{\partial\dot{q}_\alpha},~~\alpha=1,\ldots,m.$$
При этом $\mbfsansE(L)=(\mbfsansE_1(L),\ldots,\mbfsansE_m(L))$. Таким образом, уравнения Эйлера~— Лагранжа могут быть записаны в компактной форме $\mbfsansE(L)[t,q(t),\dot{q}(t)]=0$.

Зафиксируем $[t_1,t_2]\subseteq\Omega$. Определим {\em функционал действия}
\begin{equation} \label{eq1}
S[q]=\int\limits_{t_1}^{t_2} L(t,q(t),\dot{q}(t))dt
\end{equation}


Отметим, что $\mbfsansE(L)$ можно получить «проварьировав» функционал действия. Для этого рассмотрим приращение $S[q+h]-S[q]$ при условии $h(t_0)=h(t_1)=0$. Тогда, применяя формулу Тейлора, а затем интегрируя по частям, получим

$$S[q+h]-S[q]=\int\limits_{t_1}^{t_2}\left( L(t,q(t)+h(t),\dot{q}(t)+\dot{h}(t))-L(t,q(t),\dot{q}(t))\right) dt= $$
$$=\int\limits_{t_1}^{t_2}\left( \sum\limits_{\alpha=1}^m\dfrac{\partial L}{\partial q_\alpha}h_\alpha+\dfrac{\partial L}{\partial\dot{q}_\alpha}\dot{h}_\alpha\right) dt +r(h)=\int\limits_{t_1}^{t_2}\left(\sum\limits_{\alpha=1}^m\left( \dfrac{\partial L}{\partial q_\alpha}-\dfrac{d}{dt}\dfrac{\partial L}{\partial\dot{q}_\alpha}\right)h_\alpha\right)\,dt+r(h).$$
Откуда получаем
\begin{equation}\label{eq2}
S[q+h]-S[q]=\int\limits_{t_1}^{t_2}\sum\limits_{\alpha=1}^m \mbfsansE_\alpha(L)[t,q(t),\dot{q}(t)]h_\alpha(t)dt +r(h),~~\dfrac{||r||}{||h||}\rightarrow 0, ~\text{при}~h\rightarrow 0.
\end{equation}
где под $||\cdot||$ понимаем $C^1$"=норму на отрезке $[t_1,t_2]$.
Заметим, что мы фактически получили ограничение действия производной Фреше на подпространство функций с нулевыми значениями на концах отрезка $t_1$, $t_2$.

\section{Калибровка Лагранжиана}
Преобразования Лагранжиана называются {\em калибровочными}, если они не изменяют уравнений Эйлера~— Лагранжа. Простые примеры калибровочных преобразований: $L\mapsto\symup{const}\cdot L$, $L\mapsto L+\symup{const}$. Заметим, что если прибавить к лагранжиану такую функцию $f(t,q(t),\dot{q}(t))$, что $\mbfsansE(f)[t,q(t),\dot{q}(t)]\equiv 0$ $\forall t,\forall q(t)$, то такое преобразование также очевидно будет калибровочным. Имеет место следующее утверждение.


\begin{Theorem}
	$\mbfsansE(f)[t,q(t),\dot{q}(t)]\equiv 0$ $\forall t,\forall q(t)$ выполнено тогда и только тогда, когда $f(t,q(t),\dot{q}(t))$ является полной производной по $t$ от некотрой фунцкии, т.е. $$\exists g(t,q(t),\dot{q}(t)):~f=\dot{g}=\dfrac{\partial g}{\partial t}+\sum\limits_{\alpha=1}^m\dfrac{\partial g}{\partial q_\alpha}\dot{q}_\alpha+\sum\limits_{\alpha=1}^m\dfrac{\partial g}{\partial \dot{q}_\alpha}\ddot{q}_\alpha.$$
\end{Theorem}
\begin{proof}
	$ $
	\begin{enumerate}
		\item[$(\Rightarrow)$] Пусть $\mbfsansE(f)[t,q(t),\dot{q}(t)]\equiv 0$ $\forall t,\forall q(t)$. Для $\e\in[0,1]$ рассмотрим функцию $f_\e=f(t,\e q(t), \e \dot{q}(t))$. Продифференцируем ее по $\e$.
		$$\dfrac{d}{d\e}f(t,\e q(t), \e \dot{q}(t))=\sum\limits_{\alpha=1}^m\left[  q_\alpha\dfrac{\partial f}{\partial q_\alpha}\bigg|_{(t,\e q(t), \e \dot{q}(t))}+  \dot{q}_\alpha\dfrac{\partial f}{\partial \dot{q}_\alpha}\bigg|_{(t,\e q(t), \e \dot{q}(t))}\right] $$
		Ко второму слагаемому применим правило Лейбница о дифференцировании произведения.
		$$\dfrac{d}{d\e}f_\e=\sum\limits_{\alpha=1}^m\left[  q_\alpha\dfrac{\partial f}{\partial q_\alpha}\bigg|_{(t,\e q(t), \e \dot{q}(t))}-  q_\alpha\dfrac{d}{dt}\left(\dfrac{\partial f}{\partial \dot{q}_\alpha}\bigg|_{(t,\e q(t), \e \dot{q}(t))}\right) + \dfrac{d}{dt}\left(q\dfrac{\partial f}{\partial \dot{q}_\alpha}\bigg|_{(t,\e q(t), \e \dot{q}(t))} \right) \right]. $$
		$$\dfrac{d}{d\e}f_\e=\sum\limits_{\alpha=1}^m\left(q_\alpha \mbfsansE_\alpha (f)[t,\e q(t), \e \dot{q}(t)]+\dot{g}_\alpha  \right) $$
		для некоторых функциий $g_\alpha$. По условию  первое слагаемое равно нулю. Далее, обозначая сумму всех $g_\alpha$ как $g_1$,
		$$\dfrac{d}{d\e}f(t,\e q(t), \e \dot{q}(t))=\dot{g}_1(\e, t,q(t),\dot{q}(t)),$$
		$$\int\limits_0^1 \dfrac{d}{d\e}f(t,\e q(t), \e \dot{q}(t))d\e=\int\limits_0^1 \dot{g}_1(\e, t,q(t),\dot{q}(t))d\e$$
		$$f(t,q(t),\dot{q}(t))-f(t,0,0)=\dfrac{d}{dt}\int\limits_0^1 g_1(\e, t,q(t),\dot{q}(t))d\e$$
		$$f(t,q(t),\dot{q}(t))=\dfrac{d}{dt}\left(\int\limits_0^1 g_1(\e, t,q(t),\dot{q}(t))d\e+ \int_{t_0}^t f(\xi,0,0)d\xi\right) $$
		Тогда то выражение, к которому применяется оператор дифференцирования по $t$, и будет искомой функцией $g$.
		\item[$(\Leftarrow)$] Будем рассматривать $f=f(t,q(t),\dot{q}(t))$ как функцию Лагранжа. Возьмем произвольный отрезок $[t_1,t_2]\subseteq\Omega$ и рассмотрим $S$~— соответствующий функционал действия (см. \ref{eq1}). Тогда
		
		$$S=\int\limits_{t_1}^{t_2}f(t,q(t),\dot{q}(t))dt=\int\limits_{t_1}^{t_2}dg=g(t_2,q(t_2),\dot{q}(t_2))-g(t_1,q(t_1),\dot{q}(t_1))$$
 Значит, если $h(t_1)=h(t_2)=0$, то $S[q+h]-S[q]=0$ $\forall q(t),h(t)$. Поэтому из \ref{eq2} получаем, что
 $$\int\limits_{t_1}^{t_2}\sum\limits_{\alpha=1}^m \mbfsansE_\alpha(f)[t,q(t),\dot{q}(t)]h_\alpha(t)dt\equiv0~~\forall h(t). $$
 Следовательно, $\mbfsansE(f)[t,q(t),\dot{q}(t)]\equiv0$ $\forall q(t)$ $\forall t\in[t_1,t_2]$. Т.к. равенство верно на любом отрезке $[t_1,t_2]\subseteq\Omega$, то оно верно и на всем $\Omega$.
	\end{enumerate}
\end{proof}


Отметим, что сформулированные выше утверждения естественным образом переносятся на случай, когда $(q_1,\ldots,q_m)$~— локальные координатны в окрестности точки на гладком многообразии.



\addcontentsline{toc}{section}{Литература}
\begin{thebibliography}{3}


\bibitem{InGenFF}
П. Олвер, {\em Приложение групп Ли к дифференциальным уравнениям}, М:Мир, 1989


\end{thebibliography}
\end{document}
