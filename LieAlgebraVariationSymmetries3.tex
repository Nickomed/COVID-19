\documentclass[a4paper,11pt]{article}
\usepackage{luatex85}
\usepackage[babelshorthands]{polyglossia}
\usepackage[fleqn,reqno]{amsmath}
\usepackage{amsthm}
\RequirePackage[
	backend=biber,
	bibstyle=gost-numeric,
	citestyle=gost-numeric,
	%defernumbers=true,
	defernumbers=false,
	language=auto,
	autolang=langname,
	]{biblatex}
\usepackage[math-style=ISO,bold-style=upright,partial=italic]{unicode-math}
\usepackage{microtype}
\usepackage[width=16cm,height=24cm]{geometry}
\usepackage[russian]{hyperref}

\setmainlanguage{russian}

\setmainfont{Cambria}
\setsansfont{Calibri}
\setmonofont{Source Code Pro}[Scale=MatchLowercase]
\setmonofont{Source Code Pro}[Scale=MatchLowercase]
\setmathfont{Cambria Math}[sans-style=literal]
\setmathfont{XITS Math}[range={\mathscr}]

\makeatletter

\allowdisplaybreaks[4]

\def\[#1\]{\begin{align*}#1\end{align*}}
\newcommand\eqtag[1]{\refstepcounter{equation}\tag{\theequation}\label{#1}}
%\usepackage{listings}
%\usepackage{amsfonts, amssymb, amsthm, amsmath}
\usepackage[usenames]{color}
%\usepackage{fancyhdr}
\newcommand\slashfrac[2]{{#1/#2}}

%\topmargin=-10mm
%\oddsidemargin=-3mm
%\textheight=230mm
%\textwidth=170mm

%\pagestyle{fancy}
%\fancyhead{}
%\fancyfoot[C]{\thepage}
%\fancyhead[C]{Задача 6}

\def\Fi{\Phi}
\def\ZZ{\mathbb{Z}}
\def\CC{\mathbb{C}}
\def\RR{\mathbb{R}}
\def\vv{\textbf{v}}
\def\ww{\textbf{w}}
\def\uu{\textbf{u}}
\def\pr{\textbf{pr}^{(1)}}

\newtheorem*{Prb*}{Задача}
\newtheorem{Def}{Определение}

\title{Задача 6}
\author{\copyright~~Меденцов Н.В.}
%\date{undefined}

\begin{document}

\maketitle

Пусть $\vv, \ww$\ --- векторные поля на $M$. \textit{Скобка} $[\vv, \ww] = \vv
\circ \ww - \ww \circ \vv$. Для начала, покажем, что векторное пространство
векторных полей на $M$ с операцией \textit{скобка} является алгеброй Ли.
	
Свойства $(1)$ и $(2)$ проверются тривиально, исходя из определения скобки в нашем случае:
	\[
	&-[\vv, \ww] = - \vv \circ \ww + \ww \circ \vv = [\ww, \vv],\\
	&[\vv,[\uu, \ww]] + [\uu, [\ww, \vv]] + [\ww, [\vv, \uu]] ={}\\
	&\qquad
		=\vv\circ\uu\circ\ww - \vv\circ\ww\circ\uu - \uu\circ\ww\circ\vv + \ww\circ\uu\circ\vv+{}\\
	&\qquad
		+\uu\circ\ww\circ\vv - \uu\circ\vv\circ\ww - \ww\circ\vv\circ\uu + \vv\circ\ww\circ\uu+{}\\
	&\qquad
		+\ww\circ\vv\circ\uu - \ww\circ\uu\circ\vv - \vv\circ\uu\circ\ww + \uu\circ\vv\circ\ww=0.
	\]

Осталось показать, что \textit{скобка} не выводит за пределы векторного пространства:
	\[
	&\vv=\sum\limits_{i=1}^{n} \xi_i \partial_{x_i},
	\quad
	\ww=\sum\limits_{i=1}^{n} \eta_i \partial_{x_i},\\
	&\vv \circ \ww (f) = \vv(\ww(f))
		=\sum\limits_{i=1}^{n} \xi_i \partial_{x_i}(\sum\limits_{j=1}^{n} \eta_i \frac{\partial{f}}{\partial{x_i}})
		=\sum\limits_{i=1}^{n} \xi_i \left[ 
		\sum\limits_{j=1}^{n} \left(
		\frac{\partial{\eta_i}}{\partial{x_i}}
		\frac{\partial{f}}{\partial{x_j}} + 
		\frac{\partial^2{f}}{\partial{x_i}\partial{x_j}}
		\right)
		\right]={}\\
	&\qquad
		=\sum\limits_{(i, j)}\frac{\partial^2{f}}{\partial{x_i}\partial{x_j}}
		+\sum\limits_{j=1}^n\frac{\partial{f}}{\partial{x_j}} \sum\limits_{i=1}^{n} \xi_i \frac{\partial\eta_j}{\partial{x_i}}.
	\]
	
Аналогичным образом получаем, что
	\[
	\ww \circ \vv (f)=
		\sum\limits_{(i, j)} \frac{\partial^2{f}}{\partial{x_i}\partial{x_j}} +
		\sum\limits_{j=1}^{n} \frac{\partial{f}}{\partial{x_j}} \sum\limits_{i=1}^{n} \eta_i \frac{\partial\xi_j}{\partial{x_i}}.
	\]
	
Наконец, находя $[\vv, \ww](f)$, имеем
	\[
		[\vv, \ww](f) = (\vv \circ \ww - \ww \circ \vv)|_f =
		\sum\limits_{j=1}^{n} \frac{\partial{f}}{\partial{x_j}} \left(
		\sum\limits_{i=1}^{n} \xi_i \frac{\partial\eta_j}{\partial{x_i}} -
		\sum\limits_{i=1}^{n} \eta_i \frac{\partial\xi_j}{\partial{x_i}}
		\right) = 
		\sum\limits_{j=1}^{n} \partial_{x_j}(f) \zeta_j = \uu(f)
	\]

Теперь необходимо доказать, что скобка также оставляет на месте подпространство
необобщённых вариационных симметрий. Покажем, что $\forall\ \vv, \ww,$
являющихся необобщёнными вариационными симметриями лагранжевой задачи,
$\pr[\vv, \ww](L) = 0$.
% 	\frac{\partial{}}{\partial{}}
	\[
	&\pr\vv(L) = \sum_i \xi_i \frac{\partial{L}}{\partial{x_i}} + \sum_i \dot\xi_i \frac{\partial{L}}{\partial{\dot x_i}},\\
	&\pr\ww(L) = \sum_i \eta_i \frac{\partial{L}}{\partial{x_i}} + \sum_i \dot\eta_i \frac{\partial{L}}{\partial{\dot x_i}},\\
	&\pr\uu(L) = \sum_i \zeta_i \frac{\partial{L}}{\partial{x_i}}+\sum_i \dot\zeta_i \frac{\partial{L}}{\partial{\dot x_i}},\\
	&\zeta_j = \sum\limits_{i=1}^{n} \left(
		\xi_i \frac{\partial\eta_j}{\partial{x_i}} -
		\eta_i \frac{\partial\xi_j}{\partial{x_i}}
	\right),
	\quad
	\dot\zeta_j = \sum\limits_{i=1}^{n} \left(
		\xi_i\left(\frac{\partial\eta_j}{\partial{x_i}}\right)\dot{\Bigr.} +
		%\textcolor[gray]{0.6}{
			\dot\xi_i \frac{\partial\eta_j}{\partial{x_i}}
		%}
		-\eta_i\left(\frac{\partial\xi_j}{\partial{x_i}}\right)\dot{\Bigr.} -
		%\textcolor[gray]{0.6}{
			\dot\eta_i \frac{\partial\xi_j}{\partial{x_i}}
		%}
	\right).
	\]
	
Было бы приятно иметь следующий факт: $\pr\uu = [\pr\vv, \pr\ww]$, ведь тогда
	\[
	\pr\uu(L) = \pr \vv (\pr \ww(L)) - \pr \ww (\pr \vv(L)) = 0.
	\]
	
Докажем факт, описанный выше, тривиальной проверкой:
	\[
	&\pr \vv \circ \pr \ww =\\
	&\qquad
		=\sum_i \xi_i \partial_{x_i}\left(
		\sum_j \eta_j \partial_{x_j} +
		\sum_j \dot \eta_j\partial_{\dot x_j}
		\right)+
	\sum_i \dot\xi_i \partial_{\dot x_i}\left(
		\sum_j\eta_j \partial_{x_j} +
		\sum_j\dot\eta_j\partial_{\dot x_j}
		\right)={}\\
	&\qquad
		=\sum_i\xi_i \left(
		\sum_j\frac{\partial{\eta_j}}{\partial{x_i}} \partial_{x_j} + 
		\sum_j\eta_j \partial_{x_j x_i} +
		\sum_j\frac{\partial{\dot\eta_j}}{\partial{x_i}} \partial_{\dot x_j} + 
		\sum_j\dot\eta_j\partial_{\dot x_jx_i}
		\right)+{}\\
	&\qquad
		+\sum_i\dot\xi_i \left(
		\sum_j\eta_j \partial_{x_j \dot x_i} +
		\sum_j\frac{\partial\dot\eta_j}{\partial\dot x_i}\partial_{\dot x_j}+
		\sum_j\dot\eta_j\partial_{\dot x_j\dot x_i}
		\right).
	\]
	
В последнем слагаемом мы не выписали часть, содержащую
$\slashfrac{\partial{\eta_j}}{\partial{\dot x_i}}$, поскольку $\eta_j$ не зависят от
$\dot x$. При вычислении разности $\pr \vv \circ \pr \ww - \pr \ww \circ \pr
\vv$ все суммы вида $\sum_{(i, j)}$ уйдут ввиду того, что будут содержаться и в
уменьшаемом, и в вычитаемом. При выписывании результата учтём также, что
	\[
	\frac{\partial\dot\eta_j}{\partial\dot x_i}=  
  	\frac{\partial\sum_k\frac{\partial\eta_j}{x_k}\dot x_k}
  	 	 {\partial{\dot x_j}}=
  	\sum_k \frac{\partial\eta_j}{\partial x_k} \frac{\partial\dot x_k}{\dot x_j}=
  	\frac{\partial \eta_j}{\partial x_j}.
	\]
	
Остаётся выражение следующего вида:
	\[
	&[\pr\vv, \pr\ww] =
	\sum_i \xi_i \left(
		\sum_j \frac{\partial{\eta_j}}{\partial{x_i}} \partial_{x_j} + 
		\sum_j \frac{\partial{\dot\eta_j}}{\partial{x_i}} \partial_{\dot x_j}
		\right) +
	\sum_i \dot\xi_i\left(
		\sum_j \frac{\partial\dot\eta_j}{\partial{\dot x_i}}\partial_{\dot x_j}
		\right) -{}\\
	&\qquad
		-\sum_i \eta_i \left(
		\sum_j \frac{\partial{\xi_j}}{\partial{x_i}} \partial_{x_j} + 
		\sum_j \frac{\partial{\dot\xi_j}}{\partial{x_i}} \partial_{\dot x_j}
		\right) -
		\sum_i\dot\eta_i\left(
		\sum_j \frac{\partial\dot \xi_j}{\partial{\dot x_i}} \partial_{\dot x_j}
		\right)={}\\
	&\qquad
		=\sum_j \partial_{x_j} \sum\limits_{i=1}^{n} \left(
		\xi_i \frac{\partial\eta_j}{\partial{x_i}} -
		\eta_i \frac{\partial\xi_j}{\partial{x_i}}
		\right)+{}\\
	&\qquad
		+\sum_j \partial_{\dot x_j} \sum\limits_{i=1}^{n} \left(
		\xi_i\left(\frac{\partial\eta_j}{\partial{x_i}}\right)\dot{\Bigr.}+
		\dot\xi_i \frac{\partial\eta_j}{\partial{x_i}} -
		\eta_i\left(\frac{\partial\xi_j}{\partial{x_i}}\right)\dot{\Bigr.}-
		\dot\eta_i \frac{\partial\xi_j}{\partial{x_i}}
		\right).
	\]
	
Сравнивая полученное с $\pr\uu$ делаем вывод, что выражения равны, тем самым
завершая доказательство.
\end{document}

